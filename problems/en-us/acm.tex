\problemname{ACM Contest Scoring}

Our new contest submission system keeps a chronological log of all
submissions made by each team during the contest. With each entry, it
records the number of minutes into the competition at which the
submission was received, the letter that identifies the relevant
contest problem, and the result of testing the submission (designated
for the sake of this problem simply as {\tt right} or {\tt wrong}).
%
As an example, the following is a hypothetical log for a particular team:
%

\vspace{-12pt}
\begin{verbatim}
3 E right
10 A wrong
30 C wrong
50 B wrong
100 A wrong
200 A right
250 C wrong
300 D right
\end{verbatim}
\vspace{-12pt}

The rank of a team relative to others is determined by a primary and
secondary scoring measure calculated from the submission data. The
primary measure is the number of problems that were solved. The
secondary measure is based on a combination of time and penalties.
Specifically, a team's time score is equal to the sum of those
submission times that resulted in {\tt right} answers, plus a
20-minute penalty for each wrong submission of a problem that is
ultimately solved. If no problems are solved, the time measure is~$0$.

In the above example, we see that this team successfully completed
three problems: {\tt E} on their first attempt ($3$~minutes into the
contest); {\tt A} on their third attempt at that problem ($200$~minutes
into the contest); and {\tt D} on their first attempt at that problem
($300$~minutes into the contest).
%
This team's time score (including penalties) is $543$. This is
computed to include $3$~minutes for solving {\tt E}, $200$~minutes for
solving {\tt A} with an additional $40$~penalty
minutes for two earlier mistakes on that problem, and finally
$300$~minutes for solving {\tt D}.
%
Note that the team also attempted problems {\tt B} and {\tt C},
but were never successful in solving those problems, and thus received
no penalties for those attempts.

According to contest rules, after a team solves a particular problem,
any further submissions of the same problem are ignored (and thus omitted
from the log). Because times are discretized to whole minutes, there
may be more than one submission showing the same number of minutes.
In particular there could be more than one submission of the same
problem in the same minute, but they are chronological, so only the
last entry could possibly be correct.
%
As a second example, consider the following submission log:
%

\vspace{-12pt}
\begin{verbatim}
7 H right
15 B wrong
30 E wrong
35 E right
80 B wrong
80 B right
100 D wrong
100 C wrong
300 C right
300 D wrong
\end{verbatim}
\vspace{-12pt}

This team solved 4 problems, and their total time score (including
penalties) is $502$, with $7$~minutes for {\tt H}, $35+20$ for {\tt E},
$80+40$ for {\tt B}, and $300+20$ for {\tt C}.


\section*{Input}

The input contains $n$ lines for $0 \leq n \leq 100$, with each line
describing a particular log entry. A log entry has three parts:
an integer $m$, with $1 \leq m \leq 300$, designating the number of minutes at
which a submission was received, an uppercase letter designating the
problem, and either the word {\tt right} or {\tt wrong}.
The integers will be in nondecreasing order and may contain repeats.
After all the log entries is a line containing just the number $-1$.

\section*{Output}

Output two integers on a single line: the number of problems solved
and the total time measure (including penalties).
