\problemname{Eligibility}

\setlength{\columnsep}{15pt}
\illustration{0.5}{tree.png}{}%

Every year, students across the world participate in the ACM
ICPC\footnote{This may be the only problem statement in which these
  acronyms expand to Association for Computing Machinery International
  Collegiate Programming Contest.}.  In order to participate in this
contest, a student must be eligible to compete.  In this problem, you
will be given information about students and you will write a program to
determine their eligibility to participate in the ICPC.

We will start by assuming that each student meets the ``Basic Requirements'' as
specified in the ICPC rules---the student is willing to compete at the
World Finals, is a registered student with at least half-time load,
competes for only one institution in a contest year, and has not
competed in two world finals or five regional contests.

The rules to decide if a student is eligible to compete in the contest
year 2014--2015 are as follows:
\begin{enumerate}
\item if the student first began post-secondary studies in 2010 or later,
  the student is eligible;
\item if the student is born in 1991 or later, the student is eligible;
\item if none of the above applies, and the student has completed more than
 an equivalent of 8 semesters of full-time study, the student is ineligible;
\item if none of the above applies, the coach may petition for an extension
  of eligibility by providing the student's academic and work history.
\end{enumerate}
For ``equivalent of 8 semesters of full-time study,'' we consider each
semester of full-time study to be equivalent to a student completing 5
courses.  Thus, a student who has completed 41 courses or more is
considered to have more than 8 semesters of full-time study.

\section*{Input}

The input consists of a number of cases.  The first line contains a
positive integer, indicating the number of cases to follow.  Each of
the cases is specified in one line in the following format
\begin{verbatim}
  name YYYY/MM/DD YYYY/MM/DD courses
\end{verbatim}
where \verb|name| is the name of the student (up to 30 alphabetic
characters), the first date given is the date the student first began
post-secondary studies, and the second date given is the student's
date of birth.  All dates are given in the format above with 4-digit
year and 2-digit month and day.  \verb|courses| is a non-negative
integer indicating the number of courses that the student has completed.


There are at most 1\,000 cases.

\section*{Output}

For each line of output, print the student's name, followed by a space, followed
by one of the strings \verb|eligible|, \verb|ineligible|, and \verb|coach petitions|
as appropriate.
