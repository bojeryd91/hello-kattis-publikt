\problemname{Cold-puter Science}
Chicagos vintrar kan vara väldigt tuffa. Ibland sjunker temperaturen till extremt kalla nivåer, och sedan förra årets polarvirvel vill Universitetet att du tar reda på hur kalla vintern faktiskt var. De är därför intresserade av det totala antalet dagar som temperaturen låg under noll grader Celsius.

\section*{Indata}
Indatan består av två rader. Den första raden innehåller ett positivt heltal $n$ ($n$ ($1 \le n \le 100$)) som specifierar antalet temperaturmätningarna som gjorts. Den andra raden innehåller n stycken tal (varje tal är en mätning i Celsius), varje tal är separerat med ett mellanslag. Varje temperatur är ett tal $t$ ($-1\,000\,000 \le t \le 1\,000\,000$).

\section*{Utdata}
You must print a single integer: the number of temperatures strictly less than zero.
Du måste skriva ut ett enda heltal: heltalet ska vara antalet temperaturmätningar som är strikt mindre än noll.
Låt q vara en mätning, om q < 0 gäller så är q strikt mindre än noll.
